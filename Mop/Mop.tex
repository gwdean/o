% Created 2013-07-03 Wed 17:17
\documentclass[11pt]{article}
\usepackage[utf8]{inputenc}
\usepackage[T1]{fontenc}
\usepackage{graphicx}
\usepackage{longtable}
\usepackage{float}
\usepackage{wrapfig}
\usepackage{soul}
\usepackage{amssymb}
\usepackage{hyperref}


\title{Mop}
\author{Greg}
\date{03 July 2013}

\begin{document}

\maketitle

\setcounter{tocdepth}{3}
\tableofcontents
\vspace*{1cm}
\section{Mop}
\label{sec-1}

Notes on the implementation of three CLOS MOPs.

The practical end goal is to understand the CLOS
MOP it so well that I will be able to port it to 
something like the JSCL Lisp-to-Javascript compiler.

\href{https://github.com/davazp/jscl}{https://github.com/davazp/jscl}
\href{http://davazp.net/jscl/jscl.html}{http://davazp.net/jscl/jscl.html}

\subsection{MOP Briefing}
\label{sec-1.1}

The \textbf{fundamental summary} for the purposes of this
document is the description in Kiczales-1994. 

\begin{quote}
[Tiny Clos is] (a) very simple CLOS-like language,
embedded in Scheme, with a simple MOP. The features
of the default base language are:

; Classes, with instance slots, but no slot options.
; Multiple-inheritance.
; Generic functions with multi-methods and class specializers only.
; Primary methods and call-next-method; no other method combination.
; Uses Scheme's lexical scoping facilities as the class and generic
  function naming mechanism. Another way of saying this is that
  class, generic function and methods are first-class (meta)objects.

While the MOP is simple, it is essentially equal in power to both MOPs
in AMOP. This implementation is not at all optimized, but the MOP is
designed so that it can be optimized.

This MOP actually allows better optimization of slot access extensions
than those in AMOP.

In addition to calling a generic, the entry points to the default base
language are:
  
  (MAKE-CLASS list-of-superclasses list-of-slot-names)
  (MAKE-GENERIC)
  (MAKE-METHOD list-of-specializers procedure)
  (ADD-METHOD generic method)
  
  (MAKE class .initargs)
  (INITIALIZE instance initargs)      ;Add methods to this, 
                                      ;don't call it directly.
  (SLOT-REF   object slot-name)
  (SLOT-SET!  object slot-name new-value)

So, for example, one might do:
(define <position> (make-class (list <object>) (list `x `y)))
(add-method initialize
     (make-method (list <position>)
       (lambda (call-next-method pos initargs)
         (for-each (lambda (initarg-name slot-name)
                     (slot-set! pos
                                slot-name
                                (get1 initargs initarg-name 0)))
                   `(x y)
                   `(x y)))))
(set! p1 (make <position> `x 1 `y 3))

NOTE! Do not use EQUAL? to compare objects! Use EQ? or some hand
      written procedure. Objects have a pointer to their class,
      and classes are circular structures, and \ldots{}

The intospective part of the MOP looks like the following. Note that
these are ordinary procedures, not generics.

CLASS-OF

CLASS-DIRECT-SUPERS
CLASS-DIRECT-SLOTS
CLASS-CPL
CLASS-SLOTS

GENERIC-METHODS

METHOD-SPECIALIZER
METHOD-PROCEDURE

The intercessory protocol looks like (generics in uppercase):

make
  ALLOCATE-INSTANCE
  INITIALIZE              (really a base-level generic)

class initialization
  COMPUTE-CPL
  COMPUTE-SLOTS
  COMPUTE-GETTER-AND-SETTER

add-method
  COMPUTE-APPLY-GENERIC
    COMPUTE-METHODS
      COMPUTE-METHOD-MORE-SPECIFIC?
    COMPUTE-APPLY-METHODS
\end{quote}

So that's it. We will be analyzing this quote throughout.

Now from the fundamental summary we can extract the core.

A quote from ``tiny-announce.text'' associated with 1994-Kiczales states:

``The MOP in Tiny CLOS is very simple -- 8 introspective procedures
and 9 intercessory generics\ldots{}''

With this information, we can start getting a good overall view
of the MOP. 

The MOP has 17 low-level procedures/generics. That's it.

These 17 low-level objects are divided into two categories:
\begin{itemize}
\item Introspective procedures
\item Intercessory generics
\end{itemize}
\subsubsection{8 introspective procedures}
\label{sec-1.1.1}

CLASS-OF
CLASS-DIRECT-SUPERS
CLASS-DIRECT-SLOTS
CLASS-CPL
CLASS-SLOTS
GENERIC-METHODS
METHOD-SPECIALIZERS
METHOD-PROCEDURE

\subsubsection{9 intercessory generics}
\label{sec-1.1.2}

make
  ALLOCATE-INSTANCE
  INITIALIZE

class initialization
  COMPUTE-CPL
  COMPUTE-SLOTS
  COMPUTE-GETTER-AND-SETTER

add-method
  COMPUTE-APPLY-GENERIC
    COMPUTE-METHODS
      COMPUTE-METHOD-MORE-SPECIFIC?
    COMPUTE-APPLY-METHODS

\subsection{Tiny-Clos Scheme}
\label{sec-1.2}

The first version of Tiny CLOS we will look at is the
Scheme version written by Gregor Kiczales. The latest
update I have on this is 8/9/94. It was found at the
following weblink:

\href{ftp://ftp.parc.xerox.com/pub/mops/tiny/tiny-clos.scm}{ftp://ftp.parc.xerox.com/pub/mops/tiny/tiny-clos.scm}

\subsection{Tiny-Clos Common Lisp}
\label{sec-1.3}

The second version of Tiny CLOS we will look at is the
Common Lisp version written by Gregor Kiczales.

\subsection{Closette}
\label{sec-1.4}

Closette is the full blooded Common Lisp version that
is described in Gregor Kiczales, Jim des Rivieres and
Daniel G. Bobrow's classic work \emph{The Art of the Metaobject Protocol},
or \emph{AMOP} as it is known throughout the Lisp community.


\end{document}
